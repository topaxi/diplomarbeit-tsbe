\chapter{Einführung}

\label{ReportEinfuehrung}

\section{Projektcontrolling}

\subsection{Erfüllung der Ziele}

Mit der folgenden Tabelle wird aufgezeigt, welche Projektziele erfüllt,
und welche Projektziele \textbf{nicht} erfüllt wurden. Projektziele die nicht
erfüllt wurden, müssen genauer erläutert werden wieso diese nicht erfüllt wurden.

\begin{longtable}[]{@{}lp{10.5cm}c@{}}
  \toprule
  Nr.  & Zielbeschreibung                                                                                                   & Erfüllt\tabularnewline
  \toprule
  \endhead
       & \textbf{Produktziele}\tabularnewline
  \midrule
  1.1  & Besucher können im Produkt nach Konzerten suchen                                                                   & Ja\tabularnewline
  1.2  & Suchresultate können nach Musik-Genre und Ort gefiltert werden                                                     & Ja\tabularnewline
  1.3  & Besucher können Details zu einem Konzert ansehen                                                                   & Ja\tabularnewline
  1.4  & Das Produkt soll ein modernes \gls{responsive} vorweisen                                                           & Ja\tabularnewline
  1.5  & Konzerte sollen von Suchmaschinen indexiert werden können                                                          & Ja\tabularnewline
  1.6  & Benutzer können isch im Produkt registrieren                                                                       & Ja\tabularnewline
  1.7  & Benutzer können ihr Passwort nach Verlust neu setzen                                                               & \textbf{NEIN}\tabularnewline
  1.8  & Inhalte des Portals sind durch die Benutzer erfassbar und bearbeitbar                                              & Ja\tabularnewline
  1.9  & Kompatibilität mit aktuellem Google Chrome und Mozilla Firefox Browser                                             & Ja\tabularnewline
  1.10 & Konzerte können vom Produkt nach Facebook exportiert werden                                                        & Nein\tabularnewline
  1.11 & Ein angemeldeter Benutzer kann vermerken ob er einem Konzert teilnimmt                                             & Nein\tabularnewline
  1.12 & Das Produkt soll sich an die Security Best-Practices von \Gls{owasp} halten                                        & Ja\tabularnewline
  \toprule
       & \textbf{Abwicklungsziele}\tabularnewline
  \midrule
  2.1  & Das Projekt soll nach HERMES 5 unter Berücksichtigung der Richtlinien von der \Gls{tsbe} dokumentiert werden       & Ja\tabularnewline
  2.2  & Das Produkt muss bis Projektende fertiggestellt, getestet und bereit für die Einführung sein                       & Ja\tabularnewline
  2.3  & Die Technische-Umsetzung wird durch Damian Senn erstellt                                                           & Ja\tabularnewline
  2.4  & Die Kommunikation zwischen Experten und Diplomanden erfolgt wie im Projektauftrag~\ref{kommunikation} beschrieben. & Ja\tabularnewline
  2.5  & Das Projekt muss bis Ende Mai 2019 abgeschlossen sein                                                              & Ja\tabularnewline
  \bottomrule
  \caption{Controlling: Ziele}
\end{longtable}

Das Ziel 1.7 konnte nicht erfüllt werden. In Retrospektive, hätte dieses
Feature kein MUSS Kriterium sein müssen. Die verfügbare Zeit der Realisierung
wurde in wichtigere Kernkomponenten des Produktes investiert.

\clearpage
\subsection{Erfüllung der Anforderungen}

\begin{longtable}[]{@{}p{1.9cm}p{2.5cm}cp{5.5cm}cc@{}}
  \toprule
  \textbf{Feature}                & \textbf{Titel}             & \textbf{Nr.} & \textbf{Kriterium}                                                                                          & \textbf{Erfüllt}\tabularnewline
  \midrule
  \endhead
  \multirow{10}{*}{Suche}         & Suche nach Konzertname     & 1.1          & Listet alle Konzerte die Wörter der Suche im Konzertnamen beinhalten                                        & Ja                              \\ \cline{2-6}
                                  & Suche nach Konzertlocation & 1.2          & Schränkt die Such-Resultate nach gegebener Konzertlocation ein                                              & Ja                              \\ \cline{2-6}
                                  & Suche nach Ort             & 1.2          & Schränkt die Such-Resultate nach gegebenem Ort ein                                                          & Ja                              \\ \cline{2-6}
                                  & Suche nach Genre           & 1.2          & Schränkt die Such-Resultate nach gegebenem Musik-Genre ein                                                  & Ja                              \\
  \midrule
  \multirow{8}{*}{Design}         & Desktop                    & 2.1          & Alle Ansichten haben eine Desktop-Optimierte Variante                                                       & Ja                              \\ \cline{2-6}
                                  & Tablet                     & 2.2          & Alle Ansichten haben eine Tablet-Optimierte Variante                                                        & Ja                              \\ \cline{2-6}
                                  & Mobile                     & 2.3          & Alle Ansichten haben eine Mobile-Optimierte Variante                                                        & Ja                              \\ \cline{2-6}
                                  & Browser Kompatibilität     & 2.4          & Alle Ansichten müssen in aktuellem Google Chrome und Mozilla Firefox dem Grundlayout folgen                 & Ja                              \\
  \midrule
  \multirow{4}{*}{\acrshort{seo}} & Indexierbarkeit            & 3.1          & Das Produkt ist von Suchmaschinen indexierbar                                                               & Ja                              \\ \cline{2-6}
                                  & Linked Data                & 3.2          & Konzert Detailseiten sind mit dem Event-Schema\footnote{\url{https://schema.org/MusicEvent}} ausgestattet   & Ja                              \\
  \midrule
  \multirow{8}{*}{Benutzer}       & Registrierung              & 4.1          & Besucher können sich einen Benutzer registrieren, Benutzernamen und E-Mail Adressen müssen einzigartig sein & Ja                              \\ \cline{2-6}
                                  & Passwort-Vergessen         & 4.2          & Benutzer können sich einen Passwort-Reset Link anfordern                                                    & \textbf{NEIN}                   \\ \cline{2-6}
                                  & Social                     & 4.3          & Benutzer können auf Konzerten vermerken ob sie Teilnehmen oder nicht                                        & Nein                            \\
  \midrule
  \clearpage
  \multirow{6}{*}{Erfassung}      & Artist                     & 5.1          & Benutzer können Artisten mit einem Genre erfassen                                                           & \textbf{NEIN}                   \\ \cline{2-6}
                                  & Location                   & 5.2          & Benutzer können eine Konzertlocation mit Ort/Strasse erfassen                                               & \textbf{NEIN}                   \\ \cline{2-6}
                                  & Konzert                    & 5.3          & Benutzer können ein Konzert mit Konzertlocation und Artisten erfassen                                       & Ja                              \\ \cline{2-6}
                                  & Facebook                   & 5.4          & Benutzer können ein Konzert in ein Facebook-Event exportieren                                               & Nein                            \\
  \midrule
  \multirow{9}{*}{Security}       & SQL-Injection              & 6.1          & Das Produkt soll resistent gegen SQL-Injection sein                                                         & Ja                              \\ \cline{2-6}
                                  & HTML-Injection             & 6.2          & Das Produkt soll resistent gegen HTML-Injection / \acrshort{xss} sein                                       & Ja                              \\ \cline{2-6}
                                  & Passwort encryption        & 6.3          & Passwörter von Benutzer müssen mit einem sicheren Verfahren gespeichert werden                              & Ja                              \\ \cline{2-6}
                                  & Session                    & 6.4          & Session-Cookies dürfen nicht durch JavaScript ausgelesen werden                                             & Ja                              \\
  \midrule
  Performance                     & Ladezeit                   & 7.1          & Die Seitenansichten dürfen nicht länger als 6 Sekunden auf einem 3G Netz laden                              & Ja                              \\
  \midrule
  Sonstiges                       & User Tracking              & 8.1          & Benutzerverhalten soll analysiert und nachvollziehbar sein.                                                 & Nein                            \\
  \bottomrule
  \caption{Anforderungskatalog}
\end{longtable}

\clearpage
\subsubsection{Nicht erfüllte Anforderungen}

Wie bereits bei den Zielen ist auch bei den Anforderungen die Passwort-Reset (4.2)
Funktionalität nicht erfüllt.
Zusätzlich sind die Anforderungen 5.1 und 5.2 nicht erfüllt. Die beiden
Anforderungen wurden nicht umgesetzt, da in der Konzeptphase entschieden wurde,
dass die Konzertlocations nur über die Google Places \acrshort{api} ausgewählt werden können.
Somit gibt es für die Anforderung 5.2 keine Umsetzung.
Auch die Anforderung 5.1 wurde in der Konzeptphase angepasst, so wird nicht einem
Artisten das Genre sondern einem Gig die Genres angehängt. Allerdings wurde das
Feature aus Zeitgründen noch nicht umgesetzt.

\section{Wirtschaftlichkeit}

Da keine Hardware oder Ähnliches involviert ist, hat sich an der
Wirtschaftlichkeit des Projektes kaum entwas verändert.
Einige Features konnten noch nicht implementiert werden, allerdings
konnte ich aus gesundheitlichen sowie privaten Gründen weniger als
erwartet am Projekt arbeiten.
Die restlichen Features sollten in den noch übrigen 42 geplanten
Stunden implementierbar sein.
