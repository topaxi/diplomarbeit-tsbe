\chapter{Schlussbetrachtung}

\label{ReportSchlussbetrachtung}

\section{Planung}

In dieser Projektarbeit habe ich gelernt, dass die Planung das absolut
Wichtigste für ein Projekt ist.

Die Initialisierungsphase dauerte ein wenig länger als geplant, dies ist
vorallem auf die Erstellung des Projektplans zurückzuführen. So habe ich
den Plan zuerst mit der Projektmanagement Software "<GanttProject">\footnote{\url{https://www.ganttproject.biz/}}
versucht umzusetzen. Mit der Software bin ich jedoch schnell an ihre
Grenzen gestossen und konnte nur Stundenweise und nicht Wochenweise
Planen, was für eine Projektarbeit die in der Freizeit neben der
Arbeit und Schule getätigt wird eher ungünstig ist.

Glücklicherweise haben wir im Projektmanagement Unterricht bereits einen
Projektplan für eine in der Schule durchgeführte Fallstudie erstellt.
Diesen konnte ich kopieren und entsprechend diesem Projekt anpassen.

Während der Konzeptphase musste ich mit gesundheitlichen und privaten
Problemen kämpfen. Das Konzept konnte innerhalb der geplanten Zeit jedoch
trotzdem durchgeführt werden und benötigte weniger Stunden als angenommen.

Es wurde viel Zeit für die Umsetzung geplant, jedoch konnte diese Zeit gar
nicht voll und ganz in die vorhandenen Wochen investiert werden. So sind
für die Umsetzung noch rund 20\% der geplanten Zeit übrig.

Dies hat sich vor allem in den letzten zwei Wochen stark bemerkbar gemacht und
so konnten einige kleinere Muss-Kriterien nicht umgesetzt werden.

\section{Ziele und Anforderungen}

Bei der Zielsetzung und Erstellung des Anforderungskatalogs habe ich gelernt,
dass in der Phase Initialisierung die Ziele noch ein bisschen weniger
detailreich beschrieben werden können.

Einige Details wurden bei der Konkretisierung in der Konzeptphase anders als
in der Initialisierung definiert, lösen aber grundsätzlich die selben
Bedürfnisse.

\section{Realisierung}

Die Phase Realisierung war eine intensive Zeit, so musste einige Zeit aufgeholt
werden und mit Problemen mit fehlendem Know-How für das Phoenix Framework
gekämpft werden.

Das Projekt hat mir gezeigt, dass selbst wenn man ein Framework bereits für
einige andere Projekte eingesetzt hat, trotzdem noch grössere unbekannte
Bereiche auftauchen können.
