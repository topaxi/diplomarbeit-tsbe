\chapter{Studie}

\label{AppendixStudie}

\section{Zweck des Dokuments}\label{StudieZweck}

In der Studie werden die Anforderungen aufgenommen, sowie Variantenbeschriebe
für die Projektrealisierung erstellt. Die Varianten werden miteinander
verglichen und durch den Variantenentscheid wird das weitere
Vorgehen definiert.
Ausserdem werden in der Studie die Risiken und Wirtschaftlichkeit des Projekts
analysiert.

Folgende Arbeiten werden in dieser Studie abgehandelt:

\begin{itemize}
  \tightlist
  \item der Anforderungskatalog wird definiert
  \item die Evaluation der Browser Software-Technologien
  \item die Evaluation der Server Software-Technologien
  \item die Evaluation der Testing Software-Technologien
  \item eine Kostenschätzung und mögliche Wirtschaftlichkeit ausgerechnet
\end{itemize}


\section{Informationsbeschaffung}\label{informationsbeschaffung}

\begin{longtable}[]{@{}lp{10cm}@{}}
  \toprule
  Quelle                        & Beschreibung\tabularnewline
  \toprule
  Schulwissen / Berufserfahrung & Die Grundlage für die Umsetzung dieses Projekts wird durch mein existierendes Schulwissen sowie meine langjährige Berufserfahrung in der Software-Entwicklung gesetzt.\tabularnewline
  \midrule
  Internet                      & Ein Grossteil der Informationen werden heute über das Internet bezogen, für die Evaluation von Technologien und Lösungsansätzen wird einiges über das Internet recherchiert werden müssen.\tabularnewline
  \bottomrule
\end{longtable}

\clearpage
\section{Pflichtenheft/Anforderungskatalog}\label{pflichtenheftanforderungskatalog}

% TODO: Fix multirows across pages
%       https://tex.stackexchange.com/questions/79143/how-to-repeat-cell-content-on-next-page-for-longtable-using-multirow/79152
\begin{longtable}[]{@{}p{1.9cm}p{2.5cm}cp{5.5cm}cc@{}}
  \toprule
  \textbf{Feature}           & \textbf{Titel}             & \textbf{Nr.} & \textbf{Kriterium}                                                                                          & \textbf{Ziel} & \textbf{Muss}\tabularnewline
  \midrule
  \endhead
  \multirow{10}{*}{Suche}    & Suche nach Konzertname     & 1.1          & Listet alle Konzerte die Wörter der Suche im Konzertnamen beinhalten                                        & 1.1           & \textbf{Muss}                \\ \cline{2-6}
                             & Suche nach Konzertlocation & 1.2          & Schränkt die Such-Resultate nach gegebener Konzertlocation ein                                              & 1.2           & \textbf{Muss}                \\ \cline{2-6}
                             & Suche nach Ort             & 1.2          & Schränkt die Such-Resultate nach gegebenem Ort ein                                                          & 1.2           & \textbf{Muss}                \\ \cline{2-6}
                             & Suche nach Genre           & 1.2          & Schränkt die Such-Resultate nach gegebenem Musik-Genre ein                                                  & 1.2           & \textbf{Muss}                \\
  \midrule
  \multirow{8}{*}{Design}    & Desktop                    & 2.1          & Alle Ansichten haben eine Desktop-Optimierte Variante                                                       & 1.4           & \textbf{Muss}                \\ \cline{2-6}
                             & Tablet                     & 2.2          & Alle Ansichten haben eine Tablet-Optimierte Variante                                                        & 1.4           & \textbf{Muss}                \\ \cline{2-6}
                             & Mobile                     & 2.3          & Alle Ansichten haben eine Mobile-Optimierte Variante                                                        & 1.4           & \textbf{Muss}                \\ \cline{2-6}
                             & Browser Kompatibilität     & 2.4          & Alle Ansichten müssen in aktuellem Google Chrome und Mozilla Firefox dem Grundlayout folgen                 & 1.9           & \textbf{Muss}                \\
  \midrule
  \multirow{4}{*}{SEO}       & Indexierbarkeit            & 3.1          & Das Produkt ist von Suchmaschinen indexierbar                                                               & 1.5           & \textbf{Muss}                \\ \cline{2-6}
                             & Linked Data                & 3.2          & Konzert Detailseiten sind mit dem Event-Schema\footnote{https://schema.org/Event} ausgestattet              & 1.5           & \textbf{Muss}                \\
  \midrule
  \multirow{8}{*}{Benutzer}  & Registrierung              & 4.1          & Besucher können sich einen Benutzer registrieren, Benutzernamen und E-Mail Adressen müssen einzigartig sein & 1.6           & \textbf{Muss}                \\ \cline{2-6}
                             & Passwort-Vergessen         & 4.2          & Benutzer können sich einen Passwort-Reset Link anfordern                                                    & 1.7           & \textbf{Muss}                \\ \cline{2-6}
                             & Social                     & 4.3          & Benutzer können auf Konzerten vermerken ob sie Teilnehmen oder nicht                                        & 1.11          & Kann                         \\
  \midrule
  \clearpage
  \multirow{6}{*}{Erfassung} & Artist                     & 5.1          & Benutzer können Artisten mit einem Genre erfassen                                                           & 1.8           & \textbf{Muss}                \\ \cline{2-6}
                             & Location                   & 5.2          & Benutzer können eine Konzertlocation mit Ort/Strasse erfassen                                               & 1.8           & \textbf{Muss}                \\ \cline{2-6}
                             & Konzert                    & 5.3          & Benutzer können ein Konzert mit Konzertlocation und Artisten erfassen                                       & 1.8           & \textbf{Muss}                \\ \cline{2-6}
                             & Facebook                   & 5.4          & Benutzer können ein Konzert in ein Facebook-Event exportieren                                               & 1.10          & Kann                         \\
  \midrule
  \multirow{9}{*}{Security}  & SQL-Injection              & 6.1          & Das Produkt soll resistent gegen SQL-Injection sein                                                         & 1.12          & \textbf{Muss}                \\ \cline{2-6}
                             & HTML-Injection             & 6.2          & Das Produkt soll resistent gegen HTML-Injection / XSS sein                                                  & 1.12          & \textbf{Muss}                \\ \cline{2-6}
                             & Passwort encryption        & 6.3          & Passwörter von Benutzer müssen mit einem sicheren Verfahren gespeichert werden                              & 1.12          & \textbf{Muss}                \\
                             & Session                    & 6.4          & Session-Cookies dürfen nicht durch JavaScript ausgelesen werden                                             & 1.12          & Kann                         \\
  \midrule
  Performance                & Ladezeit                   & 7.1          & Die Seitenansichten dürfen nicht länger als 6 Sekunden auf einem 3G Netz laden                              &               & \textbf{Muss}                \\
  \bottomrule
\end{longtable}


\clearpage
\section{Evaluation Browser-Technologie}\label{evaluation-browser-technologie}

\begin{longtable}[]{@{}p{2cm}cp{10cm}@{}}
  \toprule
  \textbf{Kriterium} & \textbf{Gewicht} & \textbf{Abnahmekriterium}\tabularnewline
  \midrule
  \endhead
  Komplexität        & 4                & Die Technologie sollte im Rahmen der Diplomarbeit nicht eine zu hohe Komplexität vorweisen. Durch eine niedrigere Komplexität bestehen weniger Risiken dass technische Probleme auftreten werden.\tabularnewline
  \midrule
  Performance        & 4                & In den Projektzielen wurde definiert, dass die Applikation in maximal 6 Sekunden im Browser geladen sein muss. Daher ist es wichtig, dass die Technologie gute Performance Charakteristiken vorweist.\tabularnewline
  \midrule
  SEO                & 5                & Für eine öffentliche Applikation ist es unentbehrlich, dass sie indexierbar durch Suchmaschinen ist.\tabularnewline
  \midrule
  Stabilität         & 3                & \tabularnewline
  \midrule
  Testing            & 5                & \tabularnewline
  \bottomrule
\end{longtable}

\subsection{React}
\subsection{Next.js}
\subsection{SSR}

\section{Bewertungen Browser-Technologie}\label{bewertungen-browser-technologie}

\section{Entscheid Browser-Technologie}\label{entscheid-browser-technologie}

\clearpage
\section{Evaluation Server-Technologie}\label{evaluation-server-technologie}

\subsection{Node.js / koa.js}
\subsection{Elixir / Phoenix}
\subsection{Next.js}

\section{Bewertungen Server-Technologie}\label{bewertungen-server-technologie}

\section{Entscheid Server-Technologie}\label{entscheid-server-technologie}

\clearpage
\section{Evaluation Testing-Technologie}\label{evaluation-testing-technologie}

\begin{longtable}[]{@{}p{2cm}cp{10cm}@{}}
  \toprule
  \textbf{Kriterium}  & \textbf{Gewicht} & \textbf{Abnahmekriterium}\tabularnewline
  \midrule
  \endhead
  Performance         & 3                & Bei wachsender Anzahl von Tests ist es wichtig, dass die Test-Software genug skalierbar ist um Tests in parallel auszuführen.\tabularnewline
  \midrule
  Stabilität          & 5                & \tabularnewline
  \midrule
  Backend-Integration & 3                & Es ist sehr hilfreich, wenn die End-to-End Test-Software vom Server direkt ausgeführt werden. So kann gleichzeitig zum Browser-Test auch die Businesslogik getestet werden.\tabularnewline
  \midrule
  Visualtesting       & 5                & Die Technologie soll mit dem Service percy.io integrierbar sein.\tabularnewline
  \bottomrule
\end{longtable}

\subsection{Jest + Puppeteer}
\subsection{Wallaby}

\section{Bewertungen Testing-Technologie}\label{bewertungen-testing-technologie}

\section{Entscheid Testing-Technologie}\label{entscheid-testing-technologie}

\clearpage
\section{Wirtschaftlichkeit}\label{wirtschaftlichkeit}

\subsection{Break Even Analyse}\label{break-even-analyse}
