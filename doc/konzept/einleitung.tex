Durch die in der Studie gewonnenen Erkentnissen, werden in der Phase Konzept
verschiedene Teilkonzepte erstellt.

Im Teilkonzept «Portalname» wird der Name des Produktes erarbeitet.

Im Teilkonzept «Design- und Bedienkonzept» werden die Ansichten der Applikation
in Mockups umgesetzt. Es werden die Benutzer Use-Cases vom Besucher sowie der
Konzert-Erfasser aufgezeigt.

Im Teilkonzept «Softwarekonzept» werden die Datenflüsse hinter den Mockups
aufgezeigt, sowie die Datenbankstruktur aufgebaut.

Im Teilkonzept «Testkonzept» werden die einzelnen Systemtests aufgelistet sowie
ausgearbeitet wie granular welche Teile der Software getestet werden sollen.

Im letzten Teil des Konzept-Dokuments wird im Fazit dokumentiert, wie und warum
das Konzept von den vorhergehenden Phasen des Projekts abweicht.
