\chapter{Konzept}

\label{AppendixKonzept}

\section{Portalname}\label{portalname}

Der Portalname wurde in einer Brainstorming-Session von Damian Senn auf
den Namen \textbf{«Gigpillar»} festgelegt. Der Name ist angelehnt an die Werbepfeiler in
Städten, wo oft Werbeplakate für Konzerte hängen.

Die folgenden Ideen wurden in Betracht gezogen, jedoch war keine Domain mehr
verfügbar oder der Name überzeugte nicht:

\begin{itemize}
  \item{} upto.com («What are you up to?»)
  \item{} up-to.com
  \item{} uptoin.com
  \item{} gigup.com
  \item{} gigsta.com («Gigs to attend»)
  \item{} gigin.com
  \item{} gigsin.com
  \item{} gixin.com («Gigs in»)
  \item{} dualact.com («Loud act»)
  \item{} trecnoc.com («Concert» rückwärts)
\end{itemize}

\clearpage
\section{Design- und Bedienkonzept}\label{design--und-bedienkonzept}

\subsection{Mockups}

\subsubsection{Homepage}

Die Homepage ist die erste Seite, die der Besucher sieht, wenn er/sie die
Applikation direkt über \href{https://gigpillar.com/}{gigpillar.com} aufruft.
Auf den ersten Blick ist die Suche sowie ein grosses Bild (Banner) eines Gigs
zu erblicken. Weiter sind Links zu gängigen Funktionalitäten wie Gig hinzufügen
sowie das Login in einer Navigation erreichbar.

Unter dem Banner werden Gigs in nächster nähe des Besuchers aufgelistet, der
Link «change location» führt weiter zur Suchresultate Seite um den
entsprechenden Filter anzupassen.

\begin{figure}[!htb]
  \centering
  \includegraphics[width=0.95\textwidth]{mockups/homepage.png}
  \caption{Mockup: Homepage}
\end{figure}

\clearpage
\subsubsection{Suchresultate}

Auf der Suchresultate Seite sieht der Benutzer seine Suchresultate der von der
globalen Suchbox ausgelösten Suche. Die Seite bietet weitere Filter an um die
Resultate weiter einzugrenzen.\\

\noindent
Folgende Filter stehen den Benutzern zur Verfügung:

% TODO: This diverges from our search criteria described in the previous phase.

\begin{itemize}
  \tightlist{}
  \item{} Ort
  \item{} Datum von
  \item{} Datum bis
  \item{} Musik Genre
\end{itemize}

\noindent
Das Anwählen eines Suchresultates führt den Benutzer weiter zur detaillierten
Gig Ansicht.

\begin{figure}[!htb]
  \centering
  \includegraphics[width=0.95\textwidth]{mockups/search-result.png}
  \caption{Mockup: Suchresultate}
\end{figure}

\clearpage
\subsubsection{Gig Ansicht}

In der Gig Ansicht werden alle Details zu einem Event aufgelistet.

\begin{itemize}
  \tightlist{}
  \item{} Datum des Events
  \item{} Zeit wann das Event beginnt, bzw die Location die Türen öffnet
  \item{} Liste aller Künstler mit optionaler Startzeit
  \item{} Eine Beschreibung des Events
  \item{} Die Adresse der Location mit Link auf Google Maps
\end{itemize}

\noindent
Ausserdem soll es den Benutzern möglich sein, über einen «Add to my calendar»
Link das Event zu seiner Kalender-Applikation zu importieren.

\begin{figure}[!htb]
  \centering
  \includegraphics[width=0.95\textwidth]{mockups/event.png}
  \caption{Mockup: Gig Ansicht}
\end{figure}

\clearpage
\subsubsection{Gig erfassen}

Benutzer können Gigs erfassen.\\

\noindent
Folgende Daten sind für einen Gig zu erfassen:

\begin{itemize}
  \tightlist{}
  \item{} Name
  \item{} Bild \textit{(optional)}
  \item{} Location
  \item{} Datum
  \item{} Zeit
  \item{} Eine Liste von Artists mit optionaler Startzeit
  \item{} Beschreibung
  \item{} Link zum Ticketvertreiber
\end{itemize}

\begin{figure}[!htb]
  \centering
  \includegraphics[width=0.95\textwidth]{mockups/add-gig.png}
  \caption{Mockup: Gig erfassen}
\end{figure}

\clearpage
\subsubsection{Benutzerprofil}

Benutzer können ihr eigenes Profil verwalten und folgende Tätigkeiten
verrichten:

\begin{itemize}
  \tightlist{}
  \item{} Anzeigename ändern
  \item{} E-Mail Adresse ändern \textit{(mit E-Mail Bestätigung)}
  \item{} Passwort ändern \textit{(muss vorher altes Passwort bestätigen)}
  \item{} Account löschen \textit{(muss doppelt bestätigt werden!)}
\end{itemize}

\begin{figure}[!htb]
  \centering
  \includegraphics[width=0.95\textwidth]{mockups/profile.png}
  \caption{Mockup: Benutzerprofil}
\end{figure}

\clearpage
\section{Genre Filter}\label{genrefilter}

Der Genre Filter soll folgende Werte zur Verfügung stellen.

\begin{itemize}
  \item{} Alternative
  \item{} Blues
  \item{} Classical
  \item{} EDM
  \item{} Hip-Hop
  \item{} Jazz
  \item{} Metal
  \item{} Pop
  \item{} Punk
  \item{} Reggae
  \item{} Rock
\end{itemize}

\clearpage
\section{Softwarekonzept}\label{softwarekonzept}
\subsection{Datenfluss}\label{datenfluss}

\subsubsection{Homepage}\label{datenfluss-homepage}

Die Homepage zeigt den Besuchern Gigs in ihrer Nähe an, dazu muss über eine
GeoIP API die IP-Adresse des Besuchers auf ein Land zurückverfolgt werden.
Dazu wird beim ersten Besuch die GeoIP API abgefragt und das Land des Benutzers
in eine Session geschrieben. Bei weiteren Aufrufen wird das Land direkt aus der
Session bezogen.

%GigPillarWeb
%GigPillarWeb_Session
%GigPillar
%GeoIp
%
%Response = GigPillarWeb./ {
%  if (hasSession) {
%    location = GigPillarWeb_Session.getLocation()
%  } else {
%    location = GigPillarWeb_Session.createSession() {
%      location = GeoIp.getLocation(ip)
%    }
%  }
%
%  Events = GigPillar.getUpcomingEvents(location)
%}

\begin{figure}[!htb]
  \centering
  \includegraphics[width=0.95\textwidth]{konzept/datenfluss-homepage.png}
  \caption{Datenfluss: Homepage}
\end{figure}

\clearpage
\subsubsection{Suchfeld}\label{datenfluss-suchfeld}

Das globale Suchfeld hat eine Autocompletion, welche Daten direkt von
der GigPillar Applikation bezieht. Die Daten für Städtenamen wird jedoch von
einer externen Datenquelle, z.B. Google Maps, bezogen.

%GigPillarWeb
%GigPillar
%GoogleMaps
%
%Response = GigPillarWeb./searchCompletion {
%  Cities = GoogleMaps.searchCities(searchParameters)
%  Locations = GigPillar.searchLocations(searchParameters)
%  Genres = GigPillar.searchGenres(searchParameters)
%  Artists = GigPillar.searchArtists(searchParameters)
%  Events = GigPillar.searchEvents(searchParameters)
%}

\begin{figure}[!htb]
  \centering
  \includegraphics[width=0.95\textwidth]{konzept/datenfluss-suchfeld.png}
  \caption{Datenfluss: Suchfeld}
\end{figure}

\clearpage
\subsubsection{Gig erstellen - Locationfeld}\label{datenfluss-gig-erstellen-locationfeld}

Beim Erstellen eines neuen Gigs, muss eine Location zugewiesen werden. Die
Locations werden über die bereits in GigPillar erfassten Locations sowie über
eine externe Datenquelle, wie z.B. Google Maps, bezogen.

%GigPillarWeb
%GigPillar
%GoogleMaps
%
%Response = GigPillarWeb./locationCompletion {
%  Locations = GigPillar.searchLocations(searchParameters) {
%    Locations = searchLocations(searchParameters)
%    Locations = GoogleMaps.searchLocations(searchParameters)
%  }
%}

\begin{figure}[!htb]
  \centering
  \includegraphics[width=0.95\textwidth]{konzept/datenfluss-locationfeld.png}
  \caption{Datenfluss: Gig erstellen - Locationfeld}
\end{figure}

%TODO:
%\clearpage
%\subsubsection{Gig erstellen}\label{datenfluss-gigerstellen}

\clearpage
\subsubsection{Passwort-Reset}\label{datenfluss-passwort-reset}

Falls ein Benutzer sein Passwort vergessen hat, kann dieser ein neues Passwort
über die Passwort-Reset Funktion setzen. Beim Auslösen eines Passwort-Resets,
wird dem Benutzer ein E-Mail mit einem Link zugeschickt.
Der Passwort-Reset-Link führt den Benutzer auf ein Formular auf welchem er/sie
die Möglichkeit hat, ein neues Passwort zu setzen.

%GigPillarWeb
%GigPillar
%UserEmailClient
%
%Response = GigPillarWeb./passwordReset {
%  GigPillar.sendPasswordResetEmail(email) {
%    GigPillar->UserEmailClient:SMTP
%  }
%}
%
%Redirect = UserEmailClient.clickPasswordResetLink {
%  Response = GigPillarWeb./passwordRestToken
%}
%
%Response = GigPillarWeb./setPassword {
%  GigPillar.setUserPassword(currentUser, password)
%}

\begin{figure}[!htb]
  \centering
  \includegraphics[width=0.95\textwidth]{konzept/datenfluss-passwort-reset.png}
  \caption{Datenfluss: Passwort-Reset}
\end{figure}

\clearpage
\subsection{Datenbankstruktur}\label{datenbankstruktur}

\begin{figure}[!htb]
  \centering
  \includegraphics[width=0.95\textwidth]{konzept/erd.png}
  \caption{Entity Relationship Diagram}
\end{figure}

\clearpage
\section{Testkonzept}\label{testkonzept}
